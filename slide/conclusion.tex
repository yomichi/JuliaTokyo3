\begin{frame}[containsverbatim]
\frametitle{まとめ}
\begin{itemize}
  \item Julia の構文をJulia の中からいじる方法(メタプログラミング)を見てきた
    \begin{itemize}
      \item コードを自動生成することで全体のコード量や実装時間を減らせる
      \item マクロや\verb|@generated| でコンパイル時計算をしたり\verb|Memoize.jl| などで関数を自動メモ化したりすることで実行性能をあげられる(かもしれない)
      \item 自分で書く場合、どういう結果が出て欲しいかをまず考える
      \item \verb|macroexpand, dump, @show| あたりを駆使して試行錯誤
    \end{itemize}
  \item 参考資料
  \begin{itemize}
    \item \href{http://docs.julialang.org/en/latest/manual/metaprogramming/}
      {Julia 公式 Document}
      \begin{itemize}
        \item 英語が読めるならこれを読みながら手を動かせばよい
      \end{itemize}
    \item \href{http://www.asahi-net.or.jp/~kc7k-nd/onlispjhtml/}
      {On Lisp}, (著: Paul Graham, 和訳:野田開)
      \begin{itemize}
        \item Lisp 系の言語を学ぶとJulia が多大な影響を受けていることがよくわかる
        \item マクロだけじゃなくてクロージャなどの理解にも役立つ
      \end{itemize}
  \end{itemize}
\end{itemize}
\end{frame}
